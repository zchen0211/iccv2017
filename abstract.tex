\begin{abstract}
%\begin{quote}
Low-shot visual learning, the ability to recognize novel object categories from very few, or even one example, is a hallmark of human visual intelligence. Though successful on many tasks, deep learning approaches tends to be notoriously data-hungry. Recently, feature penalty regularization has been proved effective on capturing new concepts. In this work, we provide both empirical evidence and theoretical analysis on how and why these methods work. We also propose a better design of cost function with improved performance. Close scrutiny reveals the centering effect of feature representation, as well as the intrinsic connection with batch normalization. Extensive experiments on synthetic datasets, the one-shot learning benchmark ``Omniglot'', and large-scale ImageNet validate our analysis.
%\end{quote}
\end{abstract}
